%%%%%%%%%%%%%%%%%%%%%%%%%%%%%%%%%%%%%%%%%%%%%%%%%%%%%%%%%%%%%%%%%%%%%%%%%%%%%%

\documentclass{l3deliverable}

%%%%%%%%%%%%%%%%%%%%%%%%%%%%%%%%%%%%%%%%%%%%%%%%%%%%%%%%%%%%%%%%%%%%%%%%%%%%%%

\usepackage{svn-multi}
\svnid{$Id: d2.tex 2500 2011-09-21 15:56:43Z tws $}
\version{SVN Revision \svnrev~ \

Made \svnday/\svnmonth/\svnyear~ by \svnauthor}

%%%%%%%%%%%%%%%%%%%%%%%%%%%%%%%%%%%%%%%%%%%%%%%%%%%%%%%%%%%%%%%%%%%%%%%%%%%%%%
% a utility environment for formatting tasks - see document for usage
\usepackage{tabulary}
\newenvironment{PSDTask}[2]{
  \tabularx{\linewidth}{|l|X|} \hline
    \bf\itshape Task #1: & \bf\itshape #2 \\\hline
}{\endtabularx}

\newcommand{\PSDTaskComponent}[2]{\it #1: & #2 \\ \hline}
\newcommand{\PSDTaskDescription}[1]{\PSDTaskComponent{Description}{#1}}
\newcommand{\PSDTaskOutcomes}[1]{\PSDTaskComponent{Outcomes}{#1}}
\newcommand{\PSDTaskDeliverables}[1]{\PSDTaskComponent{Deliverables}{#1}}
\newcommand{\PSDTaskRisks}[1]{\PSDTaskComponent{Risk}{#1}}

%%%%%%%%%%%%%%%%%%%%%%%%%%%%%%%%%%%%%%%%%%%%%%%%%%%%%%%%%%%%%%%%%%%%%%%%%%%%%%

\usepackage{url}

%%%%%%%%%%%%%%%%%%%%%%%%%%%%%%%%%%%%%%%%%%%%%%%%%%%%%%%%%%%%%%%%%%%%%%%%%%%%%%
%% Check these macro values for appropriateness for your own document.

\title{Project Plan}

%%authors
\author{
  Author 1 \\
  Author 2 \\
  Author 3 \\
  ...}

%%release date
\date{10 January 2009}

\deliverableID{D2}
\project{PSD Group Exercise 1}
\team{X}

%%%%%%%%%%%%%%%%%%%%%%%%%%%%%%%%%%%%%%%%%%%%%%%%%%%%%%%%%%%%%%%%%%%%%%%%%%%%%%

\begin{document}

%%%%%%%%%%%%%%%%%%%%%%%%%%%%%%%%%%%%%%%%%%%%%%%%%%%%%%%%%%%%%%%%%%%%%%%%%%%%%%

\maketitle

\tableofcontents

\newpage

%%%%%%%%%%%%%%%%%%%%%%%%%%%%%%%%%%%%%%%%%%%%%%%%%%%%%%%%%%%%%%%%%%%%%%%%%%%%%%
%% Standard section for all documents

\section{Introduction}

\subsection{Identification}

\subsection{Related Documentation}

\subsection{Purpose and Description of Document}

\subsection{Document Status and Schedule}

%%%%%%%%%%%%%%%%%%%%%%%%%%%%%%%%%%%%%%%%%%%%%%%%%%%%%%%%%%%%%%%%%%%%%%%%%%%%%%

\section{Resources, Budgets, Schedules and Organisation}

%%%%%%%%%%%%%%%%%%%%%%%%%%%%%%%%%%%%%%%%%%%%%%%%%%%%%%%%%%%%%%%%%%%%%%%%%%%%%%

\subsection{Work Breakdown Structure}

Describe the logical structure for managing acquisition and
development (or relevant subsection thereof) by means of a Work
Breakdown Structure (WBS) scheme that is coordinated with the resource
allocation described in Subsection \ref{sec:allocation}. An
activities-oriented rather than an organisation- or product oriented
WBS is recommended. The level of detail given in the WBS should be
sufficient to support sound management practices.

For purposes of the WBS, identify the activities to be
undertaken. Define these in terms of a descriptive statement in
operational terms of activities and identification of the products to
be delivered or outcomes of the activity.

For each activity give: 

\begin{itemize}
\item an identifying label;
\item a descriptive statement in operational terms (what needs to be
  done);
\item identification of outcomes, including deliverables; and
\item a brief risk assessment.
\end{itemize}

For example, using the PSDTask environment (defined in this document's
header):

\begin{PSDTask}{4}{Prepare Interview Plan}
  \PSDTaskDescription{ An interview plan will be written, consisting
    of the objectives of the interview, questions to be asked,
    identification of the roles in the interview. If will then be
    reviewed, edited and approved.}%
  \PSDTaskOutcomes{Plan description document.}%
  \PSDTaskDeliverables{None}%
  \PSDTaskRisks{Questions may be in inappropriate may be too many or
    too few questions; role assignment late.}
\end{PSDTask}

%%%%%%%%%%%%%%%%%%%%%%%%%%%%%%%%%%%%%%%%%%%%%%%%%%%%%%%%%%%%%%%%%%%%%%%%%%%%%%

\subsection{Resource Estimation and Allocation to WBS\label{sec:allocation}}


The purpose of this subsection is to list and describe the resources
available to support the activities defined in the WBS. The resources
may include team members involved in the activity, roles assigned and
estimated overall effort (in person days or other appropriate
measure).

%%%%%%%%%%%%%%%%%%%%%%%%%%%%%%%%%%%%%%%%%%%%%%%%%%%%%%%%%%%%%%%%%%%%%%%%%%%%%%

\subsection{Schedules}

Present the schedules on which performance and resource planning are
based. Include a task table, Gantt and PERT charts.

%%%%%%%%%%%%%%%%%%%%%%%%%%%%%%%%%%%%%%%%%%%%%%%%%%%%%%%%%%%%%%%%%%%%%%%%%%%%%%

\subsection{Equipment, Materials, Facilities, and Other Resources}

%%%%%%%%%%%%%%%%%%%%%%%%%%%%%%%%%%%%%%%%%%%%%%%%%%%%%%%%%%%%%%%%%%%%%%%%%%%%%%

\section{Assurance Plan}

Describe the activities to be performed by for assurance of the
software and other deliverables.

Assurance activities include:
\begin{enumerate}
\item Review and acceptance testing of products
\item Verification and validation procedures
\end{enumerate}

The contents of this subsection will be explained later in the 1st
semester in the PSD lectures. For initial hand-in deadline, you may
leave this subsection blank or make your best effort to produce an
assurance plan.

%%%%%%%%%%%%%%%%%%%%%%%%%%%%%%%%%%%%%%%%%%%%%%%%%%%%%%%%%%%%%%%%%%%%%%%%%%%%%%

\section{Risk Management Plan}

The purpose of the Risk Management Plan is to identify potential risks
affecting development or acquisition, specify analysis and monitoring
methods including data to be collected, and state measures to control
or minimize the effects of the risks.

The primary topics for the plan include:

\begin{enumerate}
\item Risk Identification and Analysis
\item Monitoring
\item Avoidance
\item Mitigation
\item Review
\item List of Managed Risks 
\end{enumerate}

The contents of this subsection will be explained later in the 1st
semester in the PSD lectures. For initial hand-in deadline, you may
leave this subsection blank or make your best effort to produce an
assurance plan.

%%%%%%%%%%%%%%%%%%%%%%%%%%%%%%%%%%%%%%%%%%%%%%%%%%%%%%%%%%%%%%%%%%%%%%%%%%%%%%

\section{Configuration Management Plan}

Describe the activities and plans for configuration management to be
performed by the organization preparing this Management Plan. The
primary topics for the plan include:

\begin{enumerate}
\item Configuration management process
\item Configuration control activities
\item Configuration identification
\item Configuration change control
\end{enumerate}

The contents of this subsection will be explained later in the 1st
semester in the PSD lectures. For initial hand-in deadline, you may
leave this subsection blank or make your best effort to produce an
assurance plan.

%%%%%%%%%%%%%%%%%%%%%%%%%%%%%%%%%%%%%%%%%%%%%%%%%%%%%%%%%%%%%%%%%%%%%%%%%%%%%%

\appendix

\section{Glossary}

Including expansions of non-standard abbreviations and acronyms and
other key definitions.  You may find it useful to maintain a glossary
as a shared section amongst all your PSD documents. using the
\verb!\input{}! macro.

\section{Another appendix}

Any relevant associated documentation, e.g., a meeting plan.

\end{document}

%%%%%%%%%%%%%%%%%%%%%%%%%%%%%%%%%%%%%%%%%%%%%%%%%%%%%%%%%%%%%%%%%%%%%%%%%%%%%%

