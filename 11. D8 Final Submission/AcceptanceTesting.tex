\documentclass[12pt,a4paper,english,intoc,bibliography=totoc,index=totoc,BCOR10mm,captions=tableheading,titlepage,fleqn]{scrbook}
\usepackage{lmodern}
\renewcommand{\sfdefault}{lmss}
\renewcommand{\ttdefault}{lmtt}
\usepackage[T1]{fontenc}
\usepackage[latin9]{inputenc}
\usepackage{fancyhdr}
\pagestyle{fancy}
\setcounter{secnumdepth}{3}
\setlength{\parskip}{\medskipamount}
\setlength{\parindent}{0pt}
\usepackage{babel}
\usepackage{amsmath}
\usepackage{amssymb}
\usepackage{graphicx}
\usepackage[unicode=true,pdfusetitle,
 bookmarks=true,bookmarksnumbered=true,bookmarksopen=true,bookmarksopenlevel=1,
 breaklinks=false,pdfborder={0 0 0},backref=false,colorlinks=false]
 {hyperref}
\hypersetup{
 pdfpagelayout=OneColumn, pdfnewwindow=true, pdfstartview=XYZ, plainpages=false}

\makeatletter

\pdfpageheight\paperheight
\pdfpagewidth\paperwidth

\providecommand{\tabularnewline}{\\}

\@ifundefined{date}{}{\date{}}
\AtBeginDocument{\renewcommand{\ref}[1]{\mbox{\autoref{#1}}}}
\addto\extrasenglish{%
 \renewcommand*{\equationautorefname}[1]{}
 \renewcommand{\sectionautorefname}{sec.\negthinspace}
 \renewcommand{\subsectionautorefname}{sec.\negthinspace}
 \renewcommand{\subsubsectionautorefname}{sec.\negthinspace}
 \renewcommand{\figureautorefname}{Fig.\negthinspace}
 \renewcommand{\tableautorefname}{Tab.\negthinspace}
}

\usepackage[figure]{hypcap}

\let\myTOC\tableofcontents
\renewcommand\tableofcontents{%
  \frontmatter
  \pdfbookmark[1]{\contentsname}{}
  \myTOC
  \mainmatter }

\setkomafont{captionlabel}{\bfseries}
\setcapindent{1em}

\usepackage{calc}

\renewcommand{\chaptermark}[1]{\markboth{#1}{#1}}
\renewcommand{\sectionmark}[1]{\markright{\thesection\ #1}}

\renewcommand{\bottomfraction}{0.5}

\let\mySection\section\renewcommand{\section}{\suppressfloats[t]\mySection}

\makeatother

\begin{document}

\subject{Professional Software Development 3}


\title{Acceptance Test Plan}


\subtitle{Team L}


\author{Peeranat Fupongsiripan, Dan Tomosoiu,\\
Michael Kilian, Tony Lau,\\
Hector Grebbell}


\publishers{\includegraphics[width=6cm]{Downloads/GlasgowUniTransp}}


\dedication{This Document Describes the Acceptance Test Plan for each component
within our design of an Internship Management System for the Department
of Computing Science at the University of Glasgow. The purpose of
this acceptance test is to make sure the system developed meets its
system requirements and is suitable for all accepted use cases.}

\maketitle
Test Cases Prefixed A can be performed with an automatic test harness. Any Prefixed M must be completed manually.
\section*{Authenticator}

\begin{tabular}{lp{10cm}}
\hline 
\textbf{Test Case} & A001\tabularnewline
\hline 
\hline 
Use Case & The Authenicator is passed a valid username and password for a user
registered with the MyCampus System\tabularnewline
\hline 
Input Specification & Multiple known MyCampus Username-Password sets are fed into the authenticator.\tabularnewline
\hline 
Output Specification & For each input the authenticator should return the correct User item associated with the username given.\tabularnewline
\hline 
\end{tabular}\\

\begin{tabular}{lp{10cm}}
\hline 
\textbf{Test Case} & A002\tabularnewline
\hline 
\hline 
Use Case & The Authenicator is passed a valid username and password for a user
registered within the Internship Management System\tabularnewline
\hline 
Input Specification & Multiple known Username-Password sets from the system database are fed into the authenticator.\tabularnewline
\hline 
Output Specification & For each input the authenticator should return the correct User item associated with the username passed in.\tabularnewline
\hline 
\end{tabular}\\

\begin{tabular}{lp{10cm}}
\hline 
\textbf{Test Case} & A003\tabularnewline
\hline 
\hline 
Use Case & The Authenicator is passed a valid username with incorrect password for a user
registered with the MyCampus System\tabularnewline
\hline 
Input Specification & Multiple known MyCampus Usernames are fed into the authenticator with incorrect passwords.\tabularnewline
\hline 
Output Specification & For each input the authenticator should return null.\tabularnewline
\hline 
\end{tabular}\\

\begin{tabular}{lp{10cm}}
\hline 
\textbf{Test Case} & A004\tabularnewline
\hline 
\hline 
Use Case & The Authenicator is passed a valid username with incorrect password for a user
registered within the Internship Management System\tabularnewline
\hline 
Input Specification & Multiple known Usernames from the system database are fed into the authenticator with incorrect passwords.\tabularnewline
\hline 
Output Specification & For each input the authenticator should return null.\tabularnewline
\hline 
\end{tabular}\\

\begin{tabular}{lp{10cm}}
\hline 
\textbf{Test Case} & A005\tabularnewline
\hline 
\hline 
Use Case & The Authenicator is passed an invalid username with a correct password for a user (with a different username)
registered with the MyCampus System\tabularnewline
\hline 
Input Specification & Multiple Usernames known not to exist within the MyCampus System are fed into the authenticator with correct passwords for other users.\tabularnewline
\hline 
Output Specification & For each input the authenticator should return null.\tabularnewline
\hline 
\end{tabular}\\

\begin{tabular}{lp{10cm}}
\hline 
\textbf{Test Case} & A006\tabularnewline
\hline 
\hline 
Use Case & The Authenicator is passed a valid username with incorrect password for a user
registered within the Internship Management System\tabularnewline
\hline 
Input Specification & Multiple Usernames known not to exist within the system database are fed into the authenticator with correct passwords for other users.\tabularnewline
\hline 
Output Specification & For each input the authenticator should return null.\tabularnewline
\hline 
\end{tabular}\\

\begin{tabular}{lp{10cm}}
\hline 
\textbf{Test Case} & A006\tabularnewline
\hline 
\hline 
Use Case & The Authenicator is passed an invalid username with a password not within the Intern Management or MyCampus systems.\tabularnewline
\hline 
Input Specification & Multiple Username-Password Sets known not to exist within either system are fed into the authenticator.\tabularnewline
\hline 
Output Specification & For each input the authenticator should return null.\tabularnewline
\hline 
\end{tabular}\\

\section*{UserStore}

\begin{tabular}{lp{10cm}}
\hline 
\textbf{Test Case} & A007\tabularnewline
\hline 
\hline 
Use Case & A logged in Coordinator wishes to create a new General User Account\tabularnewline
\hline 
Input Specification & addUser() is called with the correct arguments for a specific new user. getUser() (also from this component) and authenticate() (From the authenticator) should then be called with the correct details for the new user.\tabularnewline
\hline 
Output Specification & Both getUser() and authenticate() should return the correct user item (Only valid once tests A002 and A0011).\tabularnewline
\hline 
\end{tabular}\\

\begin{tabular}{lp{10cm}}
\hline 
\textbf{Test Case} & A008\tabularnewline
\hline 
\hline 
Use Case & A logged in Coordinator wishes to create a new Student User Account\tabularnewline
\hline 
Input Specification & addStudent() is called with the correct arguments for a specific new user. getUser(), getStudent() (also from this component) and authenticate() (From the authenticator) should then be called with the correct details for the new user.\tabularnewline
\hline 
Output Specification & Both getUser() and authenticate() should return the correct user item. getStudent() should return the correct student item (Only valid once tests A002, A011 and A013).\tabularnewline
\hline 
\end{tabular}\\

\begin{tabular}{lp{10cm}}
\hline 
\textbf{Test Case} & A009\tabularnewline
\hline 
\hline 
Use Case & A logged in Coordinator wishes to create a new Employer User Account\tabularnewline
\hline 
Input Specification & addEmployer() is called with the correct arguments for a specific new user. getUser(), getEmployer() (also from this component) and authenticate() (From the authenticator) should then be called with the correct details for the new user.\tabularnewline
\hline 
Output Specification & Both getUser() and authenticate() should return the correct user item. getEmployer() should return the correct employer item (Only valid once tests A002, A011 and A014).\tabularnewline
\hline 
\end{tabular}\\

\begin{tabular}{lp{10cm}}
\hline 
\textbf{Test Case} & A010\tabularnewline
\hline 
\hline 
Use Case & A logged in Coordinator wishes to create a new Visitor User Account\tabularnewline
\hline 
Input Specification & addVisitor() is called with the correct arguments for a specific new user. getUser(), getVisitor() (also from this component) and authenticate() (From the authenticator) should then be called with the correct details for the new user.\tabularnewline
\hline 
Output Specification & Both getUser() and authenticate() should return the correct user item. getVisitor() should return the correct visitor item (Only valid once tests A002, A011 and A015).\tabularnewline
\hline 
\end{tabular}\\

\begin{tabular}{lp{10cm}}
\hline 
\textbf{Test Case} & A011\tabularnewline
\hline 
\hline 
Use Case & The System wishes to retrieve a User item from the UserStore\tabularnewline
\hline 
Input Specification & A valid username is passed to the getUser() function.\tabularnewline
\hline 
Output Specification & The correct User item is returned\tabularnewline
\hline 
\end{tabular}\\

\begin{tabular}{lp{10cm}}
\hline 
\textbf{Test Case} & A012\tabularnewline
\hline 
\hline 
Use Case & The System wishes to retrieve a Coordinator item from the UserStore\tabularnewline
\hline 
Input Specification & A valid username for the coordinator is passed to the getCoordinator() function.\tabularnewline
\hline 
Output Specification & The correct Coordinator item is returned\tabularnewline
\hline 
\end{tabular}\\

\begin{tabular}{lp{10cm}}
\hline 
\textbf{Test Case} & A013\tabularnewline
\hline 
\hline 
Use Case & The System wishes to retrieve a Student item from the UserStore\tabularnewline
\hline 
Input Specification & A valid username for the student is passed to the getStudent() function.\tabularnewline
\hline 
Output Specification & The correct Student item is returned\tabularnewline
\hline 
\end{tabular}\\

\begin{tabular}{lp{10cm}}
\hline 
\textbf{Test Case} & A014\tabularnewline
\hline 
\hline 
Use Case & The System wishes to retrieve an Employer item from the UserStore\tabularnewline
\hline 
Input Specification & A valid username for the Employer is passed to the getEmployer() function.\tabularnewline
\hline 
Output Specification & The correct Employer item is returned\tabularnewline
\hline 
\end{tabular}\\

\begin{tabular}{lp{10cm}}
\hline 
\textbf{Test Case} & A015\tabularnewline
\hline 
\hline 
Use Case & The System wishes to retrieve a Visitor item from the UserStore\tabularnewline
\hline 
Input Specification & A valid username for the Visitor is passed to the getVisitor() function.\tabularnewline
\hline 
Output Specification & The correct Visitor item is returned\tabularnewline
\hline 
\end{tabular}\\

\section*{OfferManager}

\begin{tabular}{lp{10cm}}
\hline 
\textbf{Test Case} & A016\tabularnewline
\hline 
\hline 
Use Case & A student wishes to inform the system of an accepted offer\tabularnewline
\hline 
Input Specification & notifyAcceptedOffer() is called with a range of acceptable variables. getStatus (from the placement store) should then be called on each placement\tabularnewline
\hline 
Output Specification & When getStatus is called on each placement it should return Rejected, not an error\tabularnewline
\hline 
\end{tabular}\\

\begin{tabular}{lp{10cm}}
\hline 
\textbf{Test Case} & A017\tabularnewline
\hline 
\hline 
Use Case & The Course Co-ordinator wishes to approve a placement\tabularnewline
\hline 
Input Specification & approveAcceptedOffer() should be called on a range of placements. getStatus (from the placement store) should then be called on each placement.\tabularnewline
\hline 
Output Specification & getStatus should return Accepted\tabularnewline
\hline 
\end{tabular}\\

\begin{tabular}{lp{10cm}}
\hline 
\textbf{Test Case} & A018\tabularnewline
\hline 
\hline 
Use Case & A Student wishes to inform the system of an accepted role from outside of the system.\tabularnewline
\hline 
Input Specification & createNewSelfSourcedRole() is called with a range of acceptable variables.\tabularnewline
\hline 
Output Specification & The function should return the correct role.\tabularnewline
\hline 
\end{tabular}\\

\section*{PlacementStore}

\begin{tabular}{lp{10cm}}
\hline 
\textbf{Test Case} & A019\tabularnewline
\hline 
\hline 
Use Case & The user wishes to view all placements\tabularnewline
\hline 
Input Specification & viewPlacements() should be called\tabularnewline
\hline 
Output Specification & The UI should list all placements\tabularnewline
\hline 
\end{tabular}\\

\begin{tabular}{lp{10cm}}
\hline 
\textbf{Test Case} & A020\tabularnewline
\hline 
\hline 
Use Case & A user wishes to add a placement\tabularnewline
\hline 
Input Specification & addPlacement should be called with a range of placement items\tabularnewline
\hline 
Output Specification & When eviewPlacements() is called, the new placements should also be displayed\tabularnewline
\hline 
\end{tabular}\\

\begin{tabular}{lp{10cm}}
\hline 
\textbf{Test Case} & A021\tabularnewline
\hline 
\hline 
Use Case & The coordinator wishes to remove a placement\tabularnewline
\hline 
Input Specification & removePlacement is called with the correct placement id\tabularnewline
\hline 
Output Specification & listPlacements no longer displays the placement.\tabularnewline
\hline 
\end{tabular}\\


\end{document}
