
\documentclass{article}
\usepackage[T1]{fontenc}
\usepackage[latin9]{inputenc}
\begin{document}

\section{Description Of Prototype}

\begin{flushleft}
Since we are only to build one prototype we decided to include most
of the final functionality. This allows for far more useful feedback
from the end user. There is also the added benefit over a dummy prototype
in that if there is deviation from a script, the prototype can demonstrate
accurately what is being discussed. Fundamentally however, this model
is a throw-away. Due to the nature of its construction, the additional
functionality (particularly network access, password encryption and
a GUI) needed in the final product would be far too difficult to implement.
\par\end{flushleft}

\begin{flushleft}
Data is stored within CSV files. There are 4 of these - 
\par\end{flushleft}
\begin{itemize}
\item .users.csv stores all the relevant user details for each user account.
\item .placement.csv stores details for each placement advert.
\item .studentApplication.csv stores details for applications through the
system.
\item .studentPlacement.csv s.studentPlacement.csv stores all placements,
whether for placements advertised within the system, or externally.
\end{itemize}
Upon entering the system, a user is presented with two options. Either
if they have an account, they can log in or if not they can create
an account. There are 4 account types: Co-Ordinator, Organisation
and Student. Of these only Organisation and Student can create an
account within the system since there should only ever by one Co-Ordinator
which is in effect an \textquotedblleft{}Admin\textquotedblright{}
account.

If the user opts to create an account they are asked for the necessary
details. These are then printed for them to confirm, before the account
is added to the .users.csv file. The account however is not usable
until it is verified by the co-ordinator.

If a user attempts to log in, they are asked for their username. If
this does not exist, or is not verified, an appropriate error message
is printed. If the username does exist, they are asked for their password.
If correct they are granted access to the system. Upon access, depending
on the user type they are presented with several options.


\paragraph{\textmd{For a student these are-}}
\begin{enumerate}
\item Review advertisements
\item Notify course coordinator of placement success 
\item Apply for position
\end{enumerate}

\paragraph*{\textmd{If the first of these is selected, all placement adverts
marked as visible by the co-ordinator are listed. The user is then
returned to the menu.}}


\paragraph*{\textmd{If the second option is selected, the user can either select
an advert from those in the system, or manually enter details of an
external placement.}}


\paragraph*{\textmd{If they select the third, they can apply through the system
for an advertised placement.}}


\paragraph*{\textmd{For an Organisation the options are-}}
\begin{enumerate}
\item Submit Advertisement
\item View Submitted Advertisements
\item Approve/Reject Applications
\end{enumerate}

\paragraph*{\textmd{Again the options are pretty self explanatory.}}


\paragraph*{\textmd{1 allows them to submit an advert for moderation by the co-ordinator.}}


\paragraph*{\textmd{2 allows them to see adverts they have already submitted,
and whether these have yet been made public.}}


\paragraph*{\textmd{3 allows them to approve or reject any applications to their
adverts.}}


\paragraph{\textmd{The Course co-ordinator's options are -}}
\begin{enumerate}
\item View All advertisements
\item Approve/Reject Adverts
\item Submit Advert 
\item Approve/Reject New Users 
\item View Approved Internships
\item Approve/Reject Internships 
\end{enumerate}

\paragraph{\textmd{1 lists all adverts stored in the system.}}


\paragraph{\textmd{2 allows the co-ordinator to approve or reject adverts submitted
by an organisation or academic.}}


\paragraph{\textmd{3 allows them to submit an advert for a company. These are
automatically approved and made public.}}


\paragraph{\textmd{4 allows verification or rejection of new user accounts.}}


\paragraph{\textmd{5 is a list of all current approved internships.}}


\paragraph{\textmd{6 allows approval / rejection of internships.}}

Should at any point an error occur (A common one is if one of the
files used is not present when the script is started), this is appended
to the file .errorLog.
\end{document}
