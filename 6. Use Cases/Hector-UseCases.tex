\documentclass{article}
\usepackage { usecasedescription }
%...

\begin {document}

\begin{UseCaseTemplate}
\UseCaseLabel{Account Creation}
\UseCaseDescription{It must be possibe to create accounts on the system for all users}
\UseCaseRationale{To interact on an individual basis with the system, it must have some way of telling who is using it}
\UseCasePriority{Must Have}
\UseCaseStatus{Not Implemented}
\UseCaseActors{Course Co-ordinator}
\UseCaseExtensions{- Logging into the System\\
- Automatic Enrolment for SE/ESE Students\\
- Ability for Company Accounts}
\UseCaseIncludes{}
\UseCaseConditions{\textbf{pre}\ Account ID does not already exist\\
\textbf{post}\ Account now present on system}
\UseCaseNonFunctionalRequirements{- Security (ie a Password)\\
- Graphical User Interface}
\UseCaseScenarios{\textbf{Primary:}\ Course Co-Ordinator creates an account for a student\\
\textbf{Alternative 1:}\ Automatic Enrollment for SE/ESE Students\\
\textbf{Alternative 2:}\ Course Co-Ordinator creates an account for a company}
\UseCaseRisks{}
\UseCaseUserInterface{A field based input for each value required}
\end{UseCaseTemplate}

\begin{UseCaseTemplate}
\UseCaseLabel{Automatic SE/ESE Student Enrollment}
\UseCaseDescription{The Course Co-Ordinator inputs a list of students on SE/ESE courses. The System generates a User account for each of these.}
\UseCaseRationale{All SE/ESE Students must find a placement as part of their course. Therefore it is highly likely they shall use the system (And
if the system meets all desired use-cases compulsory). By allowing automatic account creation for these students, it saves the course co-ordinator
a substantial amount of time.}
\UseCasePriority{Should Have}
\UseCaseActors{Course Co-Ordinator}
\UseCaseExtensions{}
\UseCaseIncludes{Account Creation}
\UseCaseConditions{\textbf{pre}\ Account for each student does not already exist\\
\textbf{post}\ All SE/ESE students now have an account}
\UseCaseNonFunctionalRequirements{Email Students account details\\Automatically Generate Random Passwords}
\UseCaseScenarios{\textbf{Primary:}\ The Course Co-Ordinator inputs a csv file including the necessary details for each SE/ESE student he wished to make an account for.
The accounts are then made.\\
\textbf{Alternate 1:} The Course  Co-Ordinator has recieved a large number of requests for accounts, so decides to create an input file for them all and make all the accounts at once.
}
\UseCaseRisks{If there are errors in the input file, accounts may be made with incorrect details / Not at all.}
\UseCaseUserInterface{Ability to input a file}
\end{UseCaseTemplate}

\begin{UseCaseTemplate}
\UseCaseLabel{Ability for Company Accounts}
\UseCaseDescription{A Company has an account, which allows them to submit adverts directly to the system for approval. It also allows students applications
to be forwarded directly.}
\UseCaseRationale{This saves the course co-ordinator time, since he does not have to manually enter each advert. It also means applications reach the companies more quickly.}
\UseCasePriority{Should Have}
\UseCaseActors{Co-ordinator\\Company's}
\UseCaseExtensions{}
\UseCaseIncludes{Account Creation\\Advert Submission\\Application By Students}
\UseCaseConditions{}
\UseCaseNonFunctionalRequirements{}
\UseCaseScenarios{\textbf{Primary:}\ A Company Submits an advert to the system
\textbf{Alternate 1:}\ A Student submits an application to an advert. This is then sent to the company who placed it directly.}
\UseCaseRisks{This removes the ability for the couse co-ordinator to moderate students applications}
\UseCaseUserInterface{}
\end{UseCaseTemplate}
