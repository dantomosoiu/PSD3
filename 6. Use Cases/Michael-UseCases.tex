\documentclass{article}
\usepackage { usecasedescription }
%...

\begin {document}

\begin{UseCaseTemplate}
\UseCaseLabel{Viewing of Adverts}
\UseCaseDescription{All users of the system must be able to view the advert board to review
adverts. This may only show a subset of the all adverts depending on the users status}
\UseCaseRationale{Fundamental to system description. Initial need specified in business case and exact method for viewing adverts elaborated in client interviews}
\UseCasePriority{Must Have}
\UseCaseStatus{Not Implemented}
\UseCaseActors{Organisation\\
		Student\\
		Co-ordinator}
\UseCaseExtensions{}
\UseCaseIncludes{Login}
\UseCaseConditions{None}
\UseCaseNonFunctionalRequirements{}
\UseCaseScenarios{\textbf{Primary:}\ John logs in and attempts to find a suitable placement on the advert board. He scrolls down through the advert board until he finds a 
suitable placement, or there are no adverts left to view. He closes the interface.}
\UseCaseRisks{}
\UseCaseUserInterface{}
\end{UseCaseTemplate}

\begin{UseCaseTemplate}
\UseCaseLabel{Application for placement through system}
\UseCaseDescription{User logs in and finds the advert they wish to apply for. They click onto the advert and select the option to apply through the system. They are presented
with an interface through which they make their submission using their details and being allowed to attach any additional information. They then submit the application}
\UseCaseRationale{Some smaller companies may not have their own application process so in this scenario, the client has expressed a preference for applications to be handled
through the system.}
\UseCasePriority{Should Have}
\UseCaseStatus{Not Implemented}
\UseCaseActors{Students}
\UseCaseExtensions{}
\UseCaseIncludes{Viewing of Adverts}
\UseCaseConditions{\textbf{post}\ application has been submitted and made visible to company}
\UseCaseNonFunctionalRequirements{}
\UseCaseScenarios{\textbf{Primary:}\ Karl wishes to submit an application for an advert a friend told him about. He finds the advert on the board and chooses to submit
an application. He enters all relevant extra data and sends off the application. He exits the system\
\textbf{Alternate 1:} Ian selects an advert and chooses to submit an application. When prompted to enter any extra details he accidentally clicks send. He is prompted to 
confirm he wishes to submit the application. Thanks to this warning he recovers the error and enters data. He submits the application\\
}
\UseCaseRisks{Student could send away incomplete or erroneous applications accidentally. No method exists to retrieve/edit such an application}
\UseCaseUserInterface{}
\end{UseCaseTemplate}

\begin{UseCaseTemplate}
\UseCaseLabel{Mark placements as filled}
\UseCaseDescription{User logs in to system and moves to view the advert board. They select their target application and mark it as taken by a student.}
\UseCaseRationale{There is obviously a need to communicate that a placement has been successfully filled so that other students dont waste their time making applications 
to it. However the client requested that in this situation the advert should not be removed, in case the placement becomes available again.}
\UseCasePriority{Should Have}
\UseCaseActors{Coordinator}
\UseCaseExtensions{}
\UseCaseIncludes{Viewing of Adverts}
\UseCaseConditions{
\textbf{pre}\ an advert exists for a placement which a student has filled 
\textbf{post}\ this advert is marked as taken
}
\UseCaseNonFunctionalRequirements{}
\UseCaseScenarios{\textbf{Primary:}\ Valerie logs in and has a placement she wishes to mark as fulfilled by a student. She scrolls through the advert board until she finds
the advert, selects in and chooses an option to mark it as taken.
\textbf{Alternate 1:}\ Margaret logs in and searches the advert board for a placement she wishes to mark as taken. She mistakenly selects the wrong placement and proceed to mark it as taken.
Upon realising her mistake, she reselects the advert and chooses to reopen the advert for submissions.}
\UseCaseRisks{User could mistakenly mark an advert as taken and leave the system, making the advert appear closed to students ( and therfore most likely leave a taken advert
marked as unfulfilled), failing to notice their mistake.}
\UseCaseUserInterface{}
\end{UseCaseTemplate}

\begin{UseCaseTemplate}
\UseCaseLabel{Login}
\UseCaseDescription{User logs in to system to be recognised and allow them to perform role specific actions}
\UseCaseRationale{Identified a need for different classes of user to have a different range of functionality presented. For example only the coordinator should be able to approve adverts.}
\UseCasePriority{Must Have}
\UseCaseActors{All}
\UseCaseExtensions{}
\UseCaseIncludes{}
\UseCaseConditions{\textbf{pre}\ user must have an account created in the system}
\UseCaseNonFunctionalRequirements{}
\UseCaseScenarios{\textbf{Primary:}\ Mark enters his username and password into the corresponding fields and selects login. He is logged into the system successfully.
\textbf{Alternate 1:}\ Ross makes a typing mistake whilst entering his password and attempts to login. The system reports this error and prompts him to reenter his password.
\textbf{Alternate 2:}\ Craig wishes to login to the system but has forgotten his details. He chooses to have his password sent to him. He enters his email to which the account is registered and selects send. Using
the details emailed to him he successfully logs in.}
\UseCaseRisks{}
\UseCaseUserInterface{}
\end{UseCaseTemplate}


\end{document}
