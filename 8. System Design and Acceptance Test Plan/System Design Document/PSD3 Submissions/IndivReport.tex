\documentclass{article}
\begin{document}
\title{Team Project Individual Writing Exercise}
\author{Michael Kilian}
\maketitle
\tableofcontents
\section{Introduction}
\subsection{What is an Evacuation Simulator}
A simulation can be defined as the imitation of a real-world process or system over time \cite{DiscreteEvent}.
Therefore it follows that an evacuation simulator is a system which attempts
to predict, with reasonable accuracy, the evacuation of a population from a given environment such as a building or vehicle.

Underlying any such simulator is a logical model. In a simulation model, the designer aims to
``represent the behaviour and movement observed in evacuations not only to achieve accurate results,
but to realistically represent the paths and decisions taken during an evacuation''\cite{MethodReview}. These criteria have been
achieved in our simulator by implementing a number of key features such as implementing a multi-agent architecture and the use of navigation meshes
for individual movement.

\subsection{Why Implement and Evacuation Simulator}
The task of accurately testing a building's evacuation procedure can be
both difficult and expensive. One approach is to hire members of the public
as stand-in 'evacuees' and run a mock evaluation, such as those performed as part of Forward Defensive
in preparation for the London 2012 Olympics \cite{DailyMailEvacuation}. In theory this would allow an 
appropriate expert such as a consultant from a local fire department to assess the effectiveness
of an existing plan. However such tests on public buildings can be very expensive: the cost
of hiring evacuees could be significant. Also certain aspects of evacuations, such
as testing the possibility of a crush occurring, can only really be explored by exposing participants
to real danger. Naturally this is undesirable. Finally another interesting drawback 
to this approach is that when an evacuee knows they are participating in an experiment
their behaviour is inherently different than it would be in a real evacuation. This
phenomenon is known as Evaluation Apprehension\cite{EvalApprehension}. \\

For these reasons large scale tests on a building are rarely performed, if ever.
What a simulator provides is a means for an expert to extensively test the outcomes of evacuating a location
at minimal cost. By tweaking variables in the simulation the expert can examine the probable outcome of multiple evacuations
and look for potential sources of danger in a building or evacuation procedure.

\subsection{Prerequisites For Reading This Project}
Where possible all technical concepts used within this paper will be clearly defined. However it is helpful for the reader to have a reasonable
understanding of Object-Oriented programming (preferably in Java) and of concurrent programming concepts such as multi-threading. A knowledge of 3-Dimensional
vectors and their manipulation is also beneficial.

\subsection{Previous Work}
\subsubsection{Fluid Based Systems}
These systems model crowd movement as if the crowd were a fluid \cite{WikipediaFluidMechanics}, using equations and principles taken from Physics. 
Exodus \cite{GaleaNumericalSimulation,GaleaMathModelling} is an example of such a system. There are apparent drawbacks in such a style 
of simulator. Crowds "have a choice in their direction, they have no conservation of momentum and can stop and start at will" \cite{StillCrowdDynamics}. 
These concepts are not accounted for in fluid models, and so this style of simulator is limited to estimating the movement of a crowd as a whole without
considering individual interactions.
\subsubsection{Matrix or Grid Based Systems}
In a matrix or grid based system, the floor of the environment is represented by a series of adjacent nodes, often square or hexagonal in shape. Each 
cell can represent open areas, areas blocked by a static obstacle, exits, etc. This method is becoming less common, but two formerly well know examples are 
Egress and Pedroute. It was suggested that the existing matrix-based models suffer from the
difficulties of simulating crowd cross flow and concourses; furthermore, the
assumptions employed in these models are questionable when compared with field observations \cite{StillCrowdDynamics}.
\subsubsection{Emergent Agent Based Systems}
The final class of simulator we discuss here is the emergent (agent based) simulator. In this approach the system is composed of autonomous
and interacting 'agents'. Agents interact within an environment using a defined set of simple relationships. The benefit of this approach is the introduction
of \emph{emergent} behaviour: "patterns, structures and behaviours emerge that were not explicitly programmed into the models, but arise as the 
result of agent interactions" \cite{AgentBasedTutorial}.\\
The MASSEgress project developed at Stanford University is an ongoing effort to develop a framework for the development of such systems \cite{MultiAgentFramework,IndivPseudoCode}.
It has also led to the production of at least one prototype implementing this framework. 
This project has utilised a modified version of this framework for the integration
of agent behaviour. This is discussed further in the Research and Implementation sections.

\subsection{Aims}
\subsubsection{A Multi-Agent Model of Individuals and Individual Behaviour}
Previous projects have emphasised crowd behaviour: they have modelled the population from a top down 
perspective with little consideration for the perception and interaction which an individual experiences in an evacuation.
One of the primary aims of this project is to represent a reasonably accurate model of individual behaviour.
Previous work has shown that multi-agent systems can achieve this. By taking this approach "the full effects of diversity that exists among agents in their attributes and behaviours can be observed as it
gives rises to the behaviour of the system as a whole" \cite{AgentBasedTutorial}.

\subsubsection{An Update of the Technologies Used in Previous Projects}
Many previous works in this field have made use of software packages such as Java 3D; these are now unsupported and 
would be difficult to extend in the future. Relatively new systems have been chosen for this project which we believe will remain active for an acceptable amount of time.
These are discussed in the Implementation section.

\subsubsection{Use of Navigation Meshes in Environment Modelling}
A Navigation Mesh is a graph representation of an environment in terms of a set of convex polygons which describe the `walkable' surface of an environment. This design aids agents in finding paths through large areas, whilst avoiding static obstacles in the environment.
This technique has largely been pioneered in Gaming, but is equally applicable in this setting. The benefits of a navigation mesh, or 'navmesh' is that it allows agents to move 
freely compared to other techniques such as representing the environment using a grid. A navmesh is typically combined with the powerful path finding algorithm A* to 
optimise agent movement \cite{A*Review}.

\subsection{Environment}
The Tall Ship at Riverside, Glasgow \cite{TallShipWebsite} was chosen as the environment for simulating our evacuations. It is necessary to choose an environment for testing purposes; this
is discussed further under Evaluation.\\
There are several properties of the Tall Ship which make it an interesting choice of location:
\begin{itemize}
\item{It can host 200 guests at any time, not including staff}
\item{It's structure is significantly complex to allow for the testing of most behavioural principle which we could hope to implement}
\item{No full scale evacuations have ever been carried out on the tall ship. Only mock evacuations have been performed and even these have been done at night with staff only.}
\end{itemize}

\begin{thebibliography}{99}
\bibitem{DailyMailEvacuation} \url{http://www.dailymail.co.uk/news/article-2104822/London-2012-Olympics-tube-terror-attack-drill-staged-emergency-services.html}
\bibitem{DiscreteEvent}J. Banks, J. Carson, B. Nelson, D. Nicol, (2001). \emph{Discrete-Event System Simulation}, page 3. Prentice Hall. ISBN 0-13-088702-1.
\bibitem{AgentBasedTutorial}CM Macal & MJ North (2010). \emph{Tutorial on agent-based modelling and simulation}. Palgrave Journals
\bibitem{MethodReview} S.Gwynne, E.R. Galea, P.J. Lawrence and L. Filippidis, (1999). \emph{A Review of the Methodologies Used in Evacuation Modelling}. University of Greenwich.
\bibitem{EvalApprehension} Robert Rosenthal and Ralph L. Rosnow (2009). \emph{Artifacts in Behavioral Research}, pg 212. Oxford. ISBN 978-0-19-538554-0 .
\bibitem{A*Review}Xiao Cui & Hao Shi (2011). \emph{A*-based Pathfinding in Modern Computer Games}.School of Engineering and Science, Victoria University, Melbourne, Australia.
\bibitem{TallShipWebsite}\url{www.thetallship.com}
\bibitem{GaleaNumericalSimulation}E.R. Galea (1999). \emph{The numerical simulation of aircraft evacuation and it's application to aircraft design and certification.} F.S.E.G., University of Greenwich
\bibitem{GaleaMathModelling}E.R. Galea . \emph{Use of mathematical modelling in fire safety engineering.} F.S.E.G., University of Greenwich.
\bibitem{StillCrowdDynamics}G. Keith Still (2000). \emph{Crowd Dynamics.} University of Warwick.
\bibitem{WikipediaFluidMechanics} \url{http://en.wikipedia.org/wiki/Fluid_mechanics}
\bibitem{MultiAgentFramework} Xiaoshan Pan, Charles S. Han, Ken Dauber, Kincho H. Law (2006). \emph{A multi-agent based framework for the simulation of human and social behaviours during emergency evacuations.} Stanford University.
\bibitem{IndivPseudoCode} Xiaoshan Pan, Chuck Han, Kincho H. Law, Jean-Claude Latombe (2006). \emph{A Computational Framework to Simulate Human and Social Behaviours for Egress Analysis.} Stanford University.
\end{thebibliography}
.
\end{document}
