%%%%%%%%%%%%%%%%%%%%%%%%%%%%%%%%%%%%%%%%%%%%%%%%%%%%%%%%%%%%%%%%%%%%%%%%%%%%%%

\documentclass{l3deliverable}

%%%%%%%%%%%%%%%%%%%%%%%%%%%%%%%%%%%%%%%%%%%%%%%%%%%%%%%%%%%%%%%%%%%%%%%%%%%%%%

\usepackage{graphicx}%
\usepackage[T1]{fontenc}
\usepackage[latin9]{inputenc}
%%%%%%%%%%%%%%%%%%%%%%%%%%%%%%%%%%%%%%%%%%%%%%%%%%%%%%%%%%%%%%%%%%%%%%%%%%%%%%
%% See D1 for an example of how to integrate sub version revision
%% numbers into a LaTeX document.
%


\version{1.0}


\usepackage{url}%

%%%%%%%%%%%%%%%%%%%%%%%%%%%%%%%%%%%%%%%%%%%%%%%%%%%%%%%%%%%%%%%%%%%%%%%%%%%%%%
%% Check these macro values for appropriateness for your own document.

\title{Prototyping Report}

\author{Peeranat Fupongsiripan\\
	Hector Grebbel \\
        Michael Kilian\\
        Tony Lau \\
	Dan Tomosoiu
}

\date{27 November 2012}

\deliverableID{D3}
\project{PSD3 Group Exercise 1}
\team{L}

%%%%%%%%%%%%%%%%%%%%%%%%%%%%%%%%%%%%%%%%%%%%%%%%%%%%%%%%%%%%%%%%%%%%%%%%%%%%%%

\begin{document}

%%%%%%%%%%%%%%%%%%%%%%%%%%%%%%%%%%%%%%%%%%%%%%%%%%%%%%%%%%%%%%%%%%%%%%%%%%%%%%

\maketitle

\tableofcontents

\newpage

%%%%%%%%%%%%%%%%%%%%%%%%%%%%%%%%%%%%%%%%%%%%%%%%%%%%%%%%%%%%%%%%%%%%%%%%%%%%%%
%% Standard section for all documents

\section{Introduction}
At the conclusion of an extensive period of requirements gathering in semester 1, it was decided that the team would produce a throw-away prototype to explore the
feasibility of the proposed system. The proposed features for the system were derived from a set of use cases developed by the team. 

The prototype development process consisted of four stages guided by Sommerville's prototyping process model.\cite[Page 45, Figure 2.9]{Sommerville}. These were :
\begin{enumerate}
\item{Establish Prototype Objectives}
\item{Define Prototype Functionality}
\item{Develop Prototype}
\item{Evaluate Prototype}
\end{enumerate}
This document outlines the steps taken at each of these stages and the results gained from the summative evaluation of the prototype produced.

\subsection{Purpose and Description of Document}
This document serves several purposes:
\begin{enumerate}
\item{Document the process undertaken in developing and evaluating the prototype}
\item{Outline the proposed features demonstrated in the prototype. A rationale for including each is discussed as well as any specific areas of uncertainty that the team 
wished to clarify in demonstrating the feature.}
\item{Document the features and scope of the prototype produced. The implementation details of the prototype are discussed also.}
\item{Record the results of the prototype evaluation in terms of modifications to be made to requirements and stakeholder feedback.}
\end{enumerate}

The last of these points is particularly important. The results of the evaluation must be fed into the existing requirements documentation so that this can be acted upon
as the team progresses. Therefore this document should be seen as a source of requirements until such a time when all of the information it contains has been
assessed and integrated into the team's existing requirements specification documentation.\\

\subsection{Document Status and Schedule}
For the most part this document is not intended to be a live document: no further work will be done on the development of this throw-away prototype.\\

However as discussed above the integration of the stakeholder's feedback into requirements documentation will be a process undertaken in the future. Therefore we discuss
the status of the document in terms of the evaluation results yet to be implemented.\\

A change log for this document can be found in the appendix.\\

\subsection{Related Documentation}
\begin{itemize}
\item{Requirements Specification Document}
\item{Raw Requirements List}
\end{itemize}

%%%%%%%%%%%%%%%%%%%%%%%%%%%%%%%%%%%%%%%%%%%%%%%%%%%%%%%%%%%%%%%%%%%%%%%%%%%%%%

\section{Objectives}
\subsection{Establishing Objectives and Intended Scope}
It was clearly established from the start that the objectives of the prototype were to support requirements gathering and explore the feasibility of the features already
proposed. It was not produced to prototype the user interface: the prototype uses a command line interface only.\\

The functionality represented was decided in a series of steps. Initially each team member was given a subset of the use cases
already identified and tasked with identifying any areas of uncertainty which should be explored further. Each member's contribution was posted to a wiki page on our team 
project management system to produce an unrefined list of possible prototype features.\\

At this stage it would be common to select a set of features to be implemented in the prototype, justifying the reasons for choosing these features over others. However, since
we had a relatively small set of well documented use cases, and were also behind schedule at this point in the project, it was decided instead that the team would try
to include as many of the features we had identified as possible in the time we had allocated for prototype development. High priority features or features which were sources of
significant uncertainty were to be prioritised for implementation.\\

Surprisingly, the functionality implemented successfully far exceeded initial expectations and all of the features we had identified were included. Since this was achieved with significant time to spare
we decided to also implement some features which as yet are not fully documented as part of the main requirements document which were of a low priority or optional. By demonstrating our interpretation of these requirements
we hoped to improve our understanding to the point where these features could be documented fully with use cases in the future and possibly implemented. it must be emphasised that these features are not
currently in the Requirements Specification Document; they instead come from the raw requirements list maintained by the team. These features are discussed separately; it should
not be difficult to see where these fit in with existing requirements.\\

The only use case not included in the prototype is the coordinator's ability to directly mark a placement on the advert board as filled. The team felt this was well understood enough to leave out.

\subsection{Prototype Features and Evaluation Questions}
Here we briefly outline the rationale for including each of the prototypes features and the questions we
hoped to answer in our evaluation. Note that the Generic Questions sections describes questions which are not particularly specific to any feature
which were to be asked throughout the evaluation.

\subsubsection{Utilities/Account Management}
\textbf{Login}\\
\emph{Rationale}: Core feature included in many other use cases. Fundamental to ensure
that many features work completely as intended.\
\emph{Questions}:
\begin{itemize}
\item{Are the error message used when a user enter incorrect details correct?}
\item{Is the style/form of input for logging in what the stakeholder intended?}
\end{itemize}
\\

\textbf{Account Creation}\\
\emph{Rationale}: Must have use case. Accounts must be created in order to facilitate basic interaction with the system.
While this could be ignored for prototyping purposes, there is significant uncertainty about what details are needed to
create an account and exactly how this should be done for different classes of user (i.e. student or organisation). It should be noted that the implementation differs slightly
from the requirements document: all users apply for an account and then the coordinator must log in and validate these accounts. This change was made to represent the process
of a user applying for an account by email and the coordinator creating the account.\\
\emph{Questions}:
\begin{itemize}
\item{Are the details that constitute an account the correct information needed?}
\item{Should the coordinator have to validate a newly created account? If so is the workflow we have shown the best way to
do this?}
\item{Who should be able to create new accounts? Should there be any restrictions?}
\end{itemize}

\subsubsection{Coordinator Administration}
\textbf{Advert Approval (or Rejection)}\\
\emph{Rationale}: Extremely high priority feature identified as essential from early in the project.\\
\emph{Questions}: 
\begin{itemize}
\item{How should the system handle the situation where the coordinator rejects and advert to which
a student application has already been made?}
\end{itemize}
\\

\subsubsection{Advert Viewing and Submission}
\textbf{Submission of adverts by organisations}\\
\emph{Rationale}: Extremely high priority feature identified as essential from the initial business case.\\
\emph{Questions}:
\begin{itemize}
\item{What information is necessary to an advert? In what format should this be input/processed?}
\item{Are there any missing constraints on submission?}
\item{How is a submitted advert presented to a coordinator from the completion of this process? Should the advert be held for editing or submitted directly upon completion?}
\end{itemize}
\\

\textbf{Submission of External Placements}\\
\emph{Rationale}: Reasonable priority. Submission process requires clarification.\\
\emph{Questions}:
\begin{itemize}
\item{All questions applicable to company submission of adverts are applicable here.}
\item{Should the format match that for company requirements?}
\end{itemize}
\\

\subsubsection{Advert Viewing and Applications}
\textbf{Advert Viewing}\\
\emph{Rationale}: Fairly low risk but pivotal to system. Provides a basis for student interactions with the system.\\
\emph{Questions}:
\begin{itemize}
\item{How should adverts be displayed/represented ?}
\end{itemize}
\\
\textbf{Application For a Placement Through the System}\\
\emph{Rationale}: There is a deal of uncertainty regarding exactly how this should be handled.\\
\emph{Questions}:
\begin{itemize}
\item{Should SE/ESE students be prevented from applying to placements not deemed suitable for them?}
\item{Should there be any other constraints on applications?}
\end{itemize}
\\
\textbf{Notification of Successfully Securing a Placement}\\
\emph{Rationale}: Although this feature has been established it is unclear how the notification should be represented, i.e. what
the coordinator should see and how the student issues the notification\\
\emph{Questions}:
\begin{itemize}
\item{Should there be any automatic validation to check for confirmations of securing the same placement from multiple students?}
\end{itemize}
\\
\subsection{New Features For Which Feasibility is to be Explored}
Each of these features were implemented to explore their feasibility as discussed above. For details on the proposed workflow see the implementation section.\\
\textbf{Approve/Reject Student Applications}\\
\emph{Actor}: Organisation\\
\emph{Description}: Organisations are given the ability to accept or reject students who have made an application through the system.\\
\emph{Rationale}: Since students are able to submit applications through the system it follows that it could be helpful for organisations to be able
to reply to these applications through the system as well.\\
\\
\textbf{Approve/Reject Secured Internships}
\emph{Actor}: Coordinator\\
\emph{Description}: Once a student has sent a notification that they have secured a placement, it may still be necessary for the coordinator to approve this placement or reject it
if the placement does not fit the minimum requirements to be suitable for the student. This is likely to occur if students submit an external placement which is not suitable.\\
\emph{Rationale}: Adding this feature develops a more complete view of the intended workflow which occurs when a student secures an internship.\\
\\
\textbf{View Approved Internships}\\
\emph{Actor}: Coordinator\\
\emph{Description}: The coordinator can view a list of secured placements which he has approved.\\
\emph{Rationale}: This follows naturally from the above feature to increase the visibility of the system status.\\
\\

\subsection{General Questions}
The following questions were asked of most features throughout evaluation:
\begin{itemize}
\item{Is the proposed workflow what the stakeholder expected?}
\item{Has enough relevant information been presented to the user for them to continue?}
\item{Are there any elements of the feature which the stakeholder expected but are missing?}
\end{itemize}

\subsection{Miscellaneous Questions}
These questions are related to possible additions to existing use cases or to non-functional requirements for the system:
\begin{itemize}
\item{It has been suggested that when applying to a placement through the system it should be possible for them to upload a CV. At what stage in the proposed workflow should this
be inserted? Are there any constraints on how this CV should be uploaded or formatted?}
\item{Currently data is stored using CSV files. Is there any preferred way of storing data which should be used in future?}
\end{itemize}


%%%%%%%%%%%%%%%%%%%%%%%%%%%%%%%%%%%%%%%%%%%%%%%%%%%%%%%%%%%%%%%%%%%%%%%%%%%%%%

\section{Implementation}
The prototype was implemented in \emph{bash}, a shell scripting language standard to most UNIX or GNU/Linux operating system distributions. It provided only command line interactions.\\

\begin{flushleft}
Since we are only to build one prototype we decided to include most
of the final functionality. This allows for far more useful feedback
from the end user. There is also the added benefit over a dummy prototype
in that if there is deviation from a script, the prototype can demonstrate
accurately what is being discussed. Fundamentally however, this model
is a throw-away. Due to the nature of its construction, the additional
functionality (particularly network access, password encryption and
a GUI) needed in the final product would be far too difficult to implement.
\par\end{flushleft}

\begin{flushleft}
Data is stored within CSV files. There are 4 of these - 
\par\end{flushleft}
\begin{itemize}
\item .users.csv stores all the relevant user details for each user account.
\item .placement.csv stores details for each placement advert.
\item .studentApplication.csv stores details for applications through the
system.
\item .studentPlacement.csv s.studentPlacement.csv stores all placements,
whether for placements advertised within the system, or externally.
\end{itemize}
Upon entering the system, a user is presented with two options. Either
if they have an account, they can log in or if not they can create
an account. There are 4 account types: Coordinator, Organisation
and Student. Of these only Organisation and Student can create an
account within the system since there should only ever by one coordinator
which is in effect an \textquotedblleft{}Admin\textquotedblright{}
account.

If the user opts to create an account they are asked for the necessary
details. These are then printed for them to confirm, before the account
is added to the .users.csv file. The account however is not usable
until it is verified by the coordinator.

If a user attempts to log in, they are asked for their username. If
this does not exist, or is not verified, an appropriate error message
is printed. If the username does exist, they are asked for their password.
If correct they are granted access to the system. Upon access, depending
on the user type they are presented with several options.


\paragraph{\textmd{For a student these are-}}
\begin{enumerate}
\item Review advertisements
\item Notify course coordinator of placement success 
\item Apply for position
\end{enumerate}

\paragraph*{\textmd{If the first of these is selected, all placement adverts
marked as visible by the coordinator are listed. The user is then
returned to the menu.}}


\paragraph*{\textmd{If the second option is selected, the user can either select
an advert from those in the system, or manually enter details of an
external placement.}}


\paragraph*{\textmd{If they select the third, they can apply through the system
for an advertised placement.}}


\paragraph*{\textmd{For an Organisation the options are-}}
\begin{enumerate}
\item Submit Advertisement
\item View Submitted Advertisements
\item Approve/Reject Applications
\end{enumerate}

\paragraph*{\textmd{Again the options are pretty self explanatory.}}


\paragraph*{\textmd{1 allows them to submit an advert for moderation by the coordinator.}}


\paragraph*{\textmd{2 allows them to see adverts they have already submitted,
and whether these have yet been made public.}}


\paragraph*{\textmd{3 allows them to approve or reject any applications to their
adverts.}}


\paragraph{\textmd{The Course coordinator's options are -}}
\begin{enumerate}
\item View All advertisements
\item Approve/Reject Adverts
\item Submit Advert 
\item Approve/Reject New Users 
\item View Approved Internships
\item Approve/Reject Internships 
\end{enumerate}

\paragraph{\textmd{1 lists all adverts stored in the system.}}


\paragraph{\textmd{2 allows the coordinator to approve or reject adverts submitted
by an organisation or academic.}}


\paragraph{\textmd{3 allows them to submit an advert for a company. These are
automatically approved and made public.}}


\paragraph{\textmd{4 allows verification or rejection of new user accounts.}}


\paragraph{\textmd{5 is a list of all current approved internships.}}


\paragraph{\textmd{6 allows approval / rejection of internships.}}

Should at any point an error occur (A common one is if one of the
files used is not present when the script is started), this is appended
to the file .errorLog.


\section{Evaluation and Feedback}
\subsection{How the Evaluation Was Performed}
The team was allocated a twelve minute slot in which to demonstrate as much functionality as possible to the stakeholder present at the demo.
This included the opportunity to ask the stakeholder questions regarding his/her opinion of features. We were also allowed a small window for asking
questions about the system as a whole.\\
\\
To maximise the benefit gained from the demonstration, a script was prepared beforehand detailing exactly
how the demonstration should be performed. It is not presented here as it was only intended to be used as a prompt for 
team members and is not quality assured. It can be found on the teams project management system.\\

\subsection{Feedback}
In general, the feedback received from the stakeholder was extremely positive. There were no major omissions or features in need of serious review which the stakeholder could identify.
He responded that the proposed workflow for all tasks were acceptable, including those related to new features being tested for feasibility.\\

The responses to our miscellaneous questions were as follows:
\begin{itemize}
\item{Different types of students should no be restricted from applying to any given placement. So for example a SE student
should not be restricted from applying to a placement does not fit the SESP requirements.}
\item{If a CV upload feature is added for student applications then it is not advisable to allow students to change the contents of their CV once it has been submitted to the system}
\item{The medium in which data should be stored, as well as other elements such as network access for non-university users should not be worried about for the moment: these will be discussed at a later stage.}

\end{itemize}



\appendix

%%%%%%%%%%%%%%%%%%%%%%%%%%%%%%%%%%%%%%%%%%%%%%%%%%%%%%%%%%%%%%%%%%%%%%%%%%%%%%
\section{Change Log}
\textbf{V1.0 - 28/11/2012} Initial report created. No requirements integrated.

\section{Glossary}
\textbf{SE/ESE}: Software Engineering/Electronic and Software Engineering\\
\textbf{PSD/PSD3}: Professional Software Development 3\\
\textbf{CS}: Computing Science\\
\textbf{SESP}: Software Engineering Summer Placement\\
\textbf{GUI}: Graphical User Interface\\
\textbf{CV}: Curriculum Vitae\\
\textbf{CSV}: Comma-Separated Values; a common text based file format for specifying structured data\\
\textbf{GUID}: Glasgow University Identification\\

\section{References}
\begin{thebibliography}{9}
\bibitem{Sommerville}Ian Sommerville (2011). \emph{Software Engineering, Ed. 9}. Pearson. ISBN-13: 978-0-3-705346-9.
\end{thebibliography}

%%%%%%%%%%%%%%%%%%%%%%%%%%%%%%%%%%%%%%%%%%%%%%%%%%%%%%%%%%%%%%%%%%%%%%%%%%%%%%

\end{document}

%%%%%%%%%%%%%%%%%%%%%%%%%%%%%%%%%%%%%%%%%%%%%%%%%%%%%%%%%%%%%%%%%%%%%%%%%%%%%%
