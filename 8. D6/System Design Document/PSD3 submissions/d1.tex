
%%%%%%%%%%%%%%%%%%%%%%%%%%%%%%%%%%%%%%%%%%%%%%%%%%%%%%%%%%%%%%%%%%%%%%%%%%%%%%
% This is a template for constructing your project plan document, but
% also to show the use of the l3deliverable class. Use pdflatex and
% bibtex to process the file, creating a PDF file as output (there is
% no need to use dvips when using pdflatex).
%
% Several meta data commands have been implemented to collect
% information such as deliverable identifier, project name etc (see
% below the \date command.

\documentclass{l3deliverable}

%%%%%%%%%%%%%%%%%%%%%%%%%%%%%%%%%%%%%%%%%%%%%%%%%%%%%%%%%%%%%%%%%%%%%%%%%%%%%%
% You can use the svn-multi package to automatically insert version
% control information into your document (an example of how to do this
% is shown below).  Make sure to set the 'svn:keywords' subversion
% property to 'Id' for the source file, for example, type:
%
% svn propset svn:keywords 'Id' d1.tex
%
% in the same directory as your 'd2.tex' file. 
%
% The information between the two $$ will now be updated when you next
% commit the file to your SVN repository.
%
% You can of course, just use this field to insert manual version
% information, e.g. 1.2, 1.2.1 ... instead.


\usepackage{color}
\usepackage{comment}
\usepackage[usenames,dvipsnames,svgnames,table]{xcolor}
\version{3.0}

%%%%%%%%%%%%%%%%%%%%%%%%%%%%%%%%%%%%%%%%%%%%%%%%%%%%%%%%%%%%%%%%%%%%%%%%%%%%%%

\usepackage{url}

%%%%%%%%%%%%%%%%%%%%%%%%%%%%%%%%%%%%%%%%%%%%%%%%%%%%%%%%%%%%%%%%%%%%%%%%%%%%%%
%% Check these macro values for appropriateness for your own document.

\title{Team Organisation}

%%authors
\author{
  Dan Tomosoiu \\
  Peeranat Fupongsiripan \\
  Hector Grebbell \\
  Michael Kilian \\
  Anthony Lau \\
}

%%release date 
\date{29 November 2012}

\deliverableID{D1}
\project{PSD3 Group Exercise 1}
\team{L}

%%%%%%%%%%%%%%%%%%%%%%%%%%%%%%%%%%%%%%%%%%%%%%%%%%%%%%%%%%%%%%%%%%%%%%%%%%%%%%

\begin{document}

%%%%%%%%%%%%%%%%%%%%%%%%%%%%%%%%%%%%%%%%%%%%%%%%%%%%%%%%%%%%%%%%%%%%%%%%%%%%%%

\maketitle

%%%%%%%%%%%%%%%%%%%%%%%%%%%%%%%%%%%%%%%%%%%%%%%%%%%%%%%%%%%%%%%%%%%%%%%%%%%%%%
%% Standard section for all documents

\section{Introduction}

\subsection{Identification}

This is the Management Plan of the Level 3 PSD3 Project for Team L.

\subsection{Related Documentation}

\begin{list}{}{}
\item{PSD3 Group Exercise Description: \url{http://fims.moodle.gla.ac.uk/mod/resource/view.php?id=20750}}
\item{Team L Redmine Project Management System: Login available upon request}
\item{Risk Management Plan and Risk Register: Redmine}
\item{Change Management Plan: Redmine}
\end{list}
 

\subsection{Purpose and Description of Document}
This document describes the organisation and responsibility designation within the team. Team members' individual duties are listed, ensuring the workload is split evenly. It contains 5 main sections which define, between them, the roles of the team members, the decision-making and communication strategy, mechanisms for information management and the potential risks of this particular group organisation. \\
\\
By agreeing on the spread of responsibility before the project commences, the team can avoid disagreements at a later, more time-sensitive point. The team can also attempt to see potential flaws in it's management system. By finding these now, the issues can be closer watched to prevent serious problems occurring.\\
\\
It is primarily the responsibility of the Project Manager to maintain this document.

\subsection{Document Status and Schedule}

25/09/2012 - First draft – Initial roles assigned, predicted communication tools and flaws documented.\\
{\color{red}09/10/2012} - Planned second draft – Update with any changes in role/communication structure since commencing project. Addition of any noticeable flaws.\\
{\color{red}30/10/2012} - Third draft – Any further updates as above.\\
{\color{red}27/11/2012} - Final review before submission.\\
{\color{red}29/11/2012} - Submission deadline.\\

%%%%%%%%%%%%%%%%%%%%%%%%%%%%%%%%%%%%%%%%%%%%%%%%%%%%%%%%%%%%%%%%%%%%%%%%%%%%%%

\section{Roles}

The teams organisational structure consists of the five roles described in the Administrative Programming Team\cite{steph}. These are:
\begin{itemize}
\item{\emph{Project Manager} - Michael Kilian}
\item{\emph{Toolsmith} - Peeranat Fupongsiripan}
\item{\emph{Librarian} - Hector Grebbell}
\item{\emph{Quality Assurer} - Tony Lau}
\item{\emph{Configuration Manager} - Dan Tomosoiu}

\end{itemize}

The rationale for selecting this structure is that by splitting the administrative work of the project, common and recurring tasks will be distributed evenly and each member has some degree of authority and decision making ability within the defined scope of their role. At meetings, each team member will be expected to raise any issues they have come across that is effecting their area of the project. In short, this will remove the need for any member to have an understanding of every aspect of the project.
Other roles may be included as the need for them arises and in cases where a task straddles multiple roles it may be assigned to any of the relevant members or divided between them, depending on the tasks relative size and difficulty.\\
\\
Roles may be swapped between members as time goes on. It is important for each team member to familiarise themselves with the full details of this organisational model. \\
\\
Finally, it should be noted that the module/task assigned to each team member is their responsibility. They should provide documentation, appropriate UML diagrams, test cases and an API specification for their assigned area unless otherwise agreed by the team.


%%%%%%%%%%%%%%%%%%%%%%%%%%%%%%%%%%%%%%%%%%%%%%%%%%%%%%%%%%%%%%%%%%%%%%%%%%%%%%

\section{Authority}

No member has ultimate authority. \\
\\
Team members are accountable for minor/time-sensitive decisions within their responsibility areas.\\
\\
Major issues are discussed as a group, until an agreeable outcome is reached. A member's unexplained absence at a meeting where such a decision occurs will result in them losing their vote, assuming they cannot be contacted directly before a decision is made.\\
\\
The agenda for meetings is always set by the project manager, based on what tasks need to be done with respect to the project plan. If he intends to be absent from a meeting he should make this available to the team in advance.

%%%%%%%%%%%%%%%%%%%%%%%%%%%%%%%%%%%%%%%%%%%%%%%%%%%%%%%%%%%%%%%%%%%%%%%%%%%%%%

\section{Communication}

Weekly meetings 11.00am Tuesdays and Thursdays. The time of these meetings will be divided between this project and our main team project, depending on the weekly schedule of each. The distribution of time should be negotiated in advance by the PSD3 project manager and the Team Project project manager in advance. Meetings should not exceed one hour in length.\\
\\
Informal discussion will occur through the team Facebook group.\\
\\
Redmine is our primary project management system. The wiki provided by Redmine should be used for communicating progress on distributed tasks involving multiple members contribution. For example when writing an initial requirements list, members can post new requirements gathered on the wiki so the requirements we currently have are visible and to give the team a chance as a whole to review each members contributions as they submit their own. A Trello account has been set up for the team as a back up system.
\\
If a team member cannot make a meeting, they should post a message on the Facebook Group at least 24 hours before the scheduled meeting. Any tasks which were allocated to the person who will not be present at the meeting (e.g. secretarial duties for that meeting) should be reallocated to another team member.
\\

%%%%%%%%%%%%%%%%%%%%%%%%%%%%%%%%%%%%%%%%%%%%%%%%%%%%%%%%%%%%%%%%%%%%%%%%%%%%%%

\section{Information Management}

Git is in use as a version control system. Rule regarding upload and change to the contents of the repository can be found in the change management guide: a short summary of this is available in the repository's root folder.Naturally changes logs and previous versions can be produced using Git.
\\


%%%%%%%%%%%%%%%%%%%%%%%%%%%%%%%%%%%%%%%%%%%%%%%%%%%%%%%%%%%%%%%%%%%%%%%%%%%%%%

\section{Organisational Risks}

Our biggest risk is that by dividing the work into roles, the absence of one person without effective communication could lead to disorganisation and a disproportionate workload for other members. Communicating any absences is therefore vital.
\\
Also, without one definitive leader we must all be clear on what our top-level aim is for the system. Differing interpretations of what the software is meant to do will ultimately lead to the building of the wrong system.
\\
Since we don't fully know each other's skills, there is the risk that a person may be assigned a role that they are not competent at. It may be the case that someone in the team over-estimates their ability to do a particular task. To mitigate this team members are encouraged to be as honest as possible in their competence and abilities: if you are not comfortable with a task it is by far preferable to admit this and ask for help with the task than to do it badly.\\
\\
It might also be difficult to balance the workload fairly between each member of the team.\\
\\
These are just the key risks pertinent to team organisation which have been initially identified. A more comprehensive and recent set of risks can be found in the risk register. 

%%%%%%%%%%%%%%%%%%%%%%%%%%%%%%%%%%%%%%%%%%%%%%%%%%%%%%%%%%%%%%%%%%%%%%%%%%%%%%

\appendix

%Including expansions of non-standard abbreviations and acronyms and
%other key definitions.  You may find it useful to maintain a glossary
%as a shared section amongst all your PSD documents. using the
%%API  - Application programming interface\\
PSD  - Professional Software Development\\
PSD3 - Professional Software Development 3\\
UML  - Unified Modeling Language\\

\section{Glossary} %had no idea how to do a glossary input so just hard typed it in

\item{UML - Unified Modelling Language}
\item{API - Application Programming Interface}
\item{PSD/PSD3 - Professional Software Development (3)}

%\section{Another appendix}

%Any relevant associated documentation, e.g., a meeting plan.

\begin{thebibliography}{9}
\bibitem{steph}{Stephanie Ludi. "Student Survival Guide to Managing Group Projects 2.5". \url{http://www.se.rit.edu/~sal/SEmanual/TableOfContents.html}}
\end{document}

%%%%%%%%%%%%%%%%%%%%%%%%%%%%%%%%%%%%%%%%%%%%%%%%%%%%%%%%%%%%%%%%%%%%%%%%%%%%%%
